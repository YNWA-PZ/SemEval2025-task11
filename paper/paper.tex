% This must be in the first 5 lines to tell arXiv to use pdfLaTeX, which is strongly recommended.
\pdfoutput=1
% In particular, the hyperref package requires pdfLaTeX in order to break URLs across lines.

\documentclass[11pt]{article}

% Change "review" to "final" to generate the final (sometimes called camera-ready) version.
% Change to "preprint" to generate a non-anonymous version with page numbers.
\usepackage[review]{acl}
% \usepackage{acl}

% Standard package includes
\usepackage{times}
\usepackage{latexsym}

% For proper rendering and hyphenation of words containing Latin characters (including in bib files)
\usepackage[T1]{fontenc}
% For Vietnamese characters
% \usepackage[T5]{fontenc}
% See https://www.latex-project.org/help/documentation/encguide.pdf for other character sets

% This assumes your files are encoded as UTF8
\usepackage[utf8]{inputenc}

% This is not strictly necessary, and may be commented out,
% but it will improve the layout of the manuscript,
% and will typically save some space.
\usepackage{microtype}

% This is also not strictly necessary, and may be commented out.
% However, it will improve the aesthetics of text in
% the typewriter font.
\usepackage{inconsolata}

%Including images in your LaTeX document requires adding
%additional package(s)
\usepackage{graphicx}

% If the title and author information does not fit in the area allocated, uncomment the following
%
%\setlength\titlebox{<dim>}
%
% and set <dim> to something 5cm or larger.
\usepackage{amsmath}
\usepackage{wasysym}
% \usepackage{emoji}
% \usepackage{dcolumn}  % For decimal alignment
% \usepackage{booktabs} % For professional-quality tables
% \usepackage{multirow} % For multi-row cells
\usepackage{multicol}
% \newcolumntype{d}{D{.}{.}{2.2}} % Decimal column type
% \usepackage{caption} % For captions
% \usepackage{float}
% \usepackage[utf8]{inputenc}
% \usepackage{booktabs}
% \usepackage{dcolumn,tabularx,ragged2e}
% \usepackage{siunitx}
% \usepackage{tablefootnote}
\usepackage{booktabs}
\usepackage{multirow}
\usepackage{caption}

\usepackage{longtable}
% \usepackage{rotating}

% \usepackage{natbib}
% \renewcommand\harvardurl[1]{\textbf{URL:} \url{#1}}
% \usepackage{hyperref}
% \usepackage{url}


\title{YNWA\_PZ at SemEval-2025 Task 11: Descriptive Title}

% Author information can be set in various styles:
% For several authors from the same institution:
% \author{Author 1 \and ... \and Author n \\
%         Address line \\ ... \\ Address line}
% if the names do not fit well on one line use
%         Author 1 \\ {\bf Author 2} \\ ... \\ {\bf Author n} \\
% For authors from different institutions:
% \author{Author 1 \\ Address line \\  ... \\ Address line
%         \And  ... \And
%         Author n \\ Address line \\ ... \\ Address line}
% To start a separate ``row'' of authors use \AND, as in
% \author{Author 1 \\ Address line \\  ... \\ Address line
%         \AND
%         Author 2 \\ Address line \\ ... \\ Address line \And
%         Author 3 \\ Address line \\ ... \\ Address line}

\author{First Author \\
  Affiliation / Address line 1 \\
  Affiliation / Address line 2 \\
  Affiliation / Address line 3 \\
  \texttt{email@domain} \\\And
  Second Author \\
  Affiliation / Address line 1 \\
  Affiliation / Address line 2 \\
  Affiliation / Address line 3 \\
  \texttt{email@domain} \\}

%\author{
%  \textbf{First Author\textsuperscript{1}},
%  \textbf{Second Author\textsuperscript{1,2}},
%  \textbf{Third T. Author\textsuperscript{1}},
%  \textbf{Fourth Author\textsuperscript{1}},
%\\
%  \textbf{Fifth Author\textsuperscript{1,2}},
%  \textbf{Sixth Author\textsuperscript{1}},
%  \textbf{Seventh Author\textsuperscript{1}},
%  \textbf{Eighth Author \textsuperscript{1,2,3,4}},
%\\
%  \textbf{Ninth Author\textsuperscript{1}},
%  \textbf{Tenth Author\textsuperscript{1}},
%  \textbf{Eleventh E. Author\textsuperscript{1,2,3,4,5}},
%  \textbf{Twelfth Author\textsuperscript{1}},
%\\
%  \textbf{Thirteenth Author\textsuperscript{3}},
%  \textbf{Fourteenth F. Author\textsuperscript{2,4}},
%  \textbf{Fifteenth Author\textsuperscript{1}},
%  \textbf{Sixteenth Author\textsuperscript{1}},
%\\
%  \textbf{Seventeenth S. Author\textsuperscript{4,5}},
%  \textbf{Eighteenth Author\textsuperscript{3,4}},
%  \textbf{Nineteenth N. Author\textsuperscript{2,5}},
%  \textbf{Twentieth Author\textsuperscript{1}}
%\\
%\\
%  \textsuperscript{1}Affiliation 1,
%  \textsuperscript{2}Affiliation 2,
%  \textsuperscript{3}Affiliation 3,
%  \textsuperscript{4}Affiliation 4,
%  \textsuperscript{5}Affiliation 5
%\\
%  \small{
%    \textbf{Correspondence:} \href{mailto:email@domain}{email@domain}
%  }
%}

\begin{document}
\maketitle
\begin{abstract}
  This document is a supplement to the general instructions for *ACL authors. It contains instructions for using the \LaTeX{} style files for ACL conferences.
  The document itself conforms to its own specifications, and is therefore an example of what your manuscript should look like.
  These instructions should be used both for papers submitted for review and for final versions of accepted papers.
\end{abstract}

\section{Introduction}

These instructions are for authors submitting papers to *ACL conferences using \LaTeX. They are not self-contained. All authors must follow the general instructions for *ACL proceedings,\footnote{\url{http://acl-org.github.io/ACLPUB/formatting.html}} and this document contains additional instructions for the \LaTeX{} style files.

The templates include the \LaTeX{} source of this document (\texttt{acl\_latex.tex}),
the \LaTeX{} style file used to format it (\texttt{acl.sty}),
an ACL bibliography style (\texttt{acl\_natbib.bst}),
an example bibliography (\texttt{custom.bib}),
and the bibliography for the ACL Anthology (\texttt{anthology.bib}).

\section{Introduction}
The analysis and processing of emotions from textual data have become crucial in understanding human communication across different languages and cultures. This study focuses on the detection and classification of emotions across diverse linguistic contexts, spanning regions from South America to East Asia. Our objective is to categorize emotions into key dimensions, namely sadness, anger, fear, joy, and surprise, while considering cross-lingual variations and linguistic complexities.

To address these challenges, we structure our study into three distinct tracks: (1) Track A involves binary emotion classification, determining whether a given text expresses a particular emotion; (2) Track B measures the intensity of emotions on a scale from 0 to 3, enabling a more granular understanding of emotional expressions; and (3) Track C explores cross-lingual emotion detection, facilitating insights into emotional patterns across different languages.

Understanding emotions based on textual data plays a pivotal role in various applications, including social media analysis, behavioral research, and the study of emotions' influence on social interactions. Our work contributes to the development of robust emotion recognition systems, enabling better comprehension of multilingual emotional expressions and their implications in computational linguistics.

Despite the significant advancements in emotion classification, several challenges persist. Some languages exhibit highly complex grammatical structures, making it difficult to train effective models. Additionally, the classification of emotions in low-resource languages is hindered by data scarcity and syntactic intricacies. Furthermore, certain machine learning models demonstrate suboptimal performance when applied to multilingual emotion classification, necessitating the development of novel techniques to enhance model adaptability and generalization.

To address these limitations, we present a comprehensive analysis of state-of-the-art methodologies and evaluate their effectiveness across multiple languages. Our findings highlight the critical role of language-specific preprocessing techniques, domain adaptation strategies, and transfer learning in improving multilingual emotion classification.

All code implementations, including the models and experimental setups employed in this study, are publicly available on GitHub:\footnote{\url{https://github.com/YNWA-PZ/SemEval2025-task11}}. This repository provides full documentation of our methodologies, experimental results, and final model architectures.
% \section{System Overview}

% Assume the role of an academic researcher in Computer Science. Your task is to contribute to a collaborative paper by writing a section focused on System Overview of . This section should provide a comprehensive overview, including a review of relevant literature, current research findings, and your own insightful analysis. Ensure your writing is clear, concise, and adheres to the academic standards of your field, including proper citation of sources using \citep{}. Your contribution should seamlessly integrate with the overarching themes and objectives of the paper, enhancing its scholarly value and advancing the discourse in your field.

% we preprocessed the text and for feature extraction we used diffrent models such as LSTM, multilingual Language models and LLMs.we finetuned LLMs using LORA method.
% after extracting the feature vector. we used both approaches using the embedding layer output and last hidden state of the model. we used the classifier to do the classification task such as Multi Layer Perceptron(MLP), XGBoost, Support Vector Machine(SVM) to solve multi label classification.




% mostly we used different combination of these for diffrent languages.
% - Key algorithms and modeling decisions/steps in detail.
% - The resources beyond training data (e.g., lexicons, other data,...).

% - The maths behind your model and its explanation in plain English.
% \section{System Overview}

% Our system for sentiment analysis across multiple languages is built on a series of carefully designed steps, combining preprocessing techniques, feature extraction models, and classification strategies. Below, we present the key components and decisions that shape our approach.

% \subsection{Preprocessing and Feature Extraction}
% The first step in our pipeline involves preprocessing the text data to ensure consistency and compatibility across multiple languages. This includes tasks such as tokenization, lowercasing, removing stopwords (if applicable), and handling special characters.

% For feature extraction, we experimented with a variety of models, including Long Short-Term Memory networks (LSTMs), multilingual language models (e.g., mBERT, XLM-R), and large language models (LLMs) such as GPT-based architectures. These models were chosen for their ability to capture semantic meaning across different languages effectively.

% After generating the feature representations, we explored two primary approaches:
% \begin{itemize}
%     \item Using the embedding layer output directly as feature vectors.
%     \item Leveraging the last hidden state of the model to capture the contextualized representation of the input text.
% \end{itemize}

% \subsection{Classification Step}
% Once the feature vectors were extracted, we focused on solving the multi-label classification problem. We investigated a range of classifiers to determine the most effective approach for this task. The models we used include:
% \begin{itemize}
%     \item \textbf{Multi-Layer Perceptron (MLP):} A neural network-based approach to model complex relationships between features and output labels.
%     \item \textbf{XGBoost:} A gradient boosting framework that performs well on tabular data and can handle multi-label tasks efficiently.
%     \item \textbf{Support Vector Machine (SVM):} A traditional machine learning classifier, which we adapted to handle multi-label outputs.
% \end{itemize}

% By combining these classifiers with the extracted features, we aimed to achieve high performance across different languages and sentiment labels.

% \subsection{Fine-Tuning LLMs with LoRA}
% To further enhance the performance of the large language models (LLMs), we applied the Low-Rank Adaptation (LoRA) fine-tuning method. LoRA allows for efficient fine-tuning by freezing the pre-trained model weights and introducing trainable low-rank matrices. This approach significantly reduces the computational cost of fine-tuning while maintaining strong performance. Using LoRA, we adapted the LLMs to our specific multi-label sentiment analysis task, ensuring that they could better handle the nuances of different languages and sentiment categories.

% \subsection{Mathematical Foundation in Plain English}
% At the core of our system lies the mathematical representation of text data. In simple terms:
% \begin{itemize}
%     \item The text is first converted into numerical representations (vectors) by the feature extraction models. These vectors capture the meaning and context of the words in the text.
%     \item The classifiers then map these vectors to the output labels. For multi-label classification, this means predicting multiple sentiment categories (e.g., positive, negative, neutral) for a single input text. Each label is treated as a binary decision (present or absent), and the model optimizes for all labels simultaneously.
% \end{itemize}

% The fine-tuning process (e.g., LoRA) refines the weights of the models to align better with our specific dataset, ensuring accurate predictions across multiple languages.

% \subsection{Model Variants and Experiments}
% Throughout our experimentation, we tested several variants of the system:
% \begin{itemize}
%     \item \textbf{Feature extraction models:} Comparing LSTMs, multilingual models (like mBERT), and LLMs to determine their effectiveness in capturing cross-lingual sentiment.
%     \item \textbf{Classification approaches:} Evaluating the performance of MLP, XGBoost, and SVM with different feature representations.
%     \item \textbf{Fine-tuning strategies:} Exploring the impact of LoRA versus full fine-tuning on the LLMs.
% \end{itemize}

% By systematically evaluating these variants, we identified the optimal configuration for our multi-label, multi-language sentiment analysis task.
% \section{Methodology}
\section{System Overview}

In this section, we present a comprehensive overview of our system for multi-label text classification, which integrates various deep learning architectures and machine learning classifiers. The system follows a pipeline that includes text preprocessing, feature extraction using neural network models, and classification through different machine learning algorithms.

\subsection{Preprocessing}

The preprocessing pipeline involves multiple steps to clean and standardize the text data. Initially, all text is converted to lowercase and unnecessary whitespace is removed. Then, special characters, URLs, and emojis are filtered out using regular expressions. Each emoji is replaced by its corresponding title (e.g., \smiley → "smiling face"). Tokenization is performed using language-specific tokenizers to ensure optimal segmentation. Finally, stopwords are removed, and text normalization techniques, such as stemming or lemmatization, are applied where appropriate.

\subsection{Feature Extraction}

To extract features, we employed a diverse range of models, including Long Short-Term Memory (LSTM) networks, multilingual language models, and Large Language Models (LLMs). The LLMs were fine-tuned using Low-Rank Adaptation (LoRA) \citep{hu2021lora}, a parameter-efficient tuning method that facilitates task-specific adaptation while maintaining computational efficiency.

The extracted feature vectors were derived using two distinct approaches. The first approach utilized the output from the embedding layer of the models, which captures contextual word representations in a lower-dimensional vector space. The second approach involved extracting the final hidden state of the neural network, which encapsulates high-level semantic information of the text.

\subsection{Classification Approach}

Following feature extraction, we applied multiple classification algorithms to perform the multi-label classification task. One of the classifiers used was the Multi-Layer Perceptron (MLP), a feedforward artificial neural network capable of modeling complex relationships between the extracted features and the target labels. Additionally, the system employed XGBoost, a gradient boosting framework renowned for its effectiveness in structured data classification \citep{chen2016xgboost}. Furthermore, Support Vector Machines (SVMs) were utilized as a classification method due to their ability to operate effectively in high-dimensional feature spaces by identifying optimal hyperplanes for classification \citep{cortes1995support}.

By combining these techniques, the proposed system captures rich linguistic representations of the input text and achieves accurate multi-label classification. The experimental setup and evaluation metrics used to validate the effectiveness of this approach are discussed in subsequent sections.
\section{Preamble}

The first line of the file must be
\begin{quote}
  \begin{verbatim}
\documentclass[11pt]{article}
\end{verbatim}
\end{quote}

To load the style file in the review version:
\begin{quote}
  \begin{verbatim}
\usepackage[review]{acl}
\end{verbatim}
\end{quote}
For the final version, omit the \verb|review| option:
\begin{quote}
  \begin{verbatim}
\usepackage{acl}
\end{verbatim}
\end{quote}

To use Times Roman, put the following in the preamble:
\begin{quote}
  \begin{verbatim}
\usepackage{times}
\end{verbatim}
\end{quote}
(Alternatives like txfonts or newtx are also acceptable.)

Please see the \LaTeX{} source of this document for comments on other packages that may be useful.

Set the title and author using \verb|\title| and \verb|\author|. Within the author list, format multiple authors using \verb|\and| and \verb|\And| and \verb|\AND|; please see the \LaTeX{} source for examples.

By default, the box containing the title and author names is set to the minimum of 5 cm. If you need more space, include the following in the preamble:
\begin{quote}
  \begin{verbatim}
\setlength\titlebox{<dim>}
\end{verbatim}
\end{quote}
where \verb|<dim>| is replaced with a length. Do not set this length smaller than 5 cm.

\section{Document Body}

\subsection{Footnotes}

Footnotes are inserted with the \verb|\footnote| command.\footnote{This is a footnote.}

\subsection{Tables and figures}

See Table~\ref{tab:accents} for an example of a table and its caption.
\textbf{Do not override the default caption sizes.}

\begin{table}
  \centering
  \begin{tabular}{lc}
    \hline
    \textbf{Command} & \textbf{Output} \\
    \hline
    \verb|{\"a}|     & {\"a}           \\
    \verb|{\^e}|     & {\^e}           \\
    \verb|{\`i}|     & {\`i}           \\
    \verb|{\.I}|     & {\.I}           \\
    \verb|{\o}|      & {\o}            \\
    \verb|{\'u}|     & {\'u}           \\
    \verb|{\aa}|     & {\aa}           \\\hline
  \end{tabular}
  \begin{tabular}{lc}
    \hline
    \textbf{Command} & \textbf{Output} \\
    \hline
    \verb|{\c c}|    & {\c c}          \\
    \verb|{\u g}|    & {\u g}          \\
    \verb|{\l}|      & {\l}            \\
    \verb|{\~n}|     & {\~n}           \\
    \verb|{\H o}|    & {\H o}          \\
    \verb|{\v r}|    & {\v r}          \\
    \verb|{\ss}|     & {\ss}           \\
    \hline
  \end{tabular}
  \caption{Example commands for accented characters, to be used in, \emph{e.g.}, Bib\TeX{} entries.}
  \label{tab:accents}
\end{table}

As much as possible, fonts in figures should conform
to the document fonts. See Figure~\ref{fig:experiments} for an example of a figure and its caption.

Using the \verb|graphicx| package graphics files can be included within figure
environment at an appropriate point within the text.
The \verb|graphicx| package supports various optional arguments to control the
appearance of the figure.
You must include it explicitly in the \LaTeX{} preamble (after the
\verb|\documentclass| declaration and before \verb|\begin{document}|) using
\verb|\usepackage{graphicx}|.

\begin{figure}[t]
  \includegraphics[width=\columnwidth]{example-image-golden}
  \caption{A figure with a caption that runs for more than one line.
    Example image is usually available through the \texttt{mwe} package
    without even mentioning it in the preamble.}
  \label{fig:experiments}
\end{figure}

\begin{figure*}[t]
  \includegraphics[width=0.48\linewidth]{example-image-a} \hfill
  \includegraphics[width=0.48\linewidth]{example-image-b}
  \caption {A minimal working example to demonstrate how to place
    two images side-by-side.}
\end{figure*}

\subsection{Hyperlinks}

Users of older versions of \LaTeX{} may encounter the following error during compilation:
\begin{quote}
  \verb|\pdfendlink| ended up in different nesting level than \verb|\pdfstartlink|.
\end{quote}
This happens when pdf\LaTeX{} is used and a citation splits across a page boundary. The best way to fix this is to upgrade \LaTeX{} to 2018-12-01 or later.

\subsection{Citations}

\begin{table*}
  \centering
  \begin{tabular}{lll}
    \hline
    \textbf{Output}           & \textbf{natbib command} & \textbf{ACL only command} \\
    \hline
    \citep{Gusfield:97}       & \verb|\citep|           &                           \\
    \citealp{Gusfield:97}     & \verb|\citealp|         &                           \\
    \citet{Gusfield:97}       & \verb|\citet|           &                           \\
    \citeyearpar{Gusfield:97} & \verb|\citeyearpar|     &                           \\
    \citeposs{Gusfield:97}    &                         & \verb|\citeposs|          \\
    \hline
  \end{tabular}
  \caption{\label{citation-guide}
    Citation commands supported by the style file.
    The style is based on the natbib package and supports all natbib citation commands.
    It also supports commands defined in previous ACL style files for compatibility.
  }
\end{table*}

Table~\ref{citation-guide} shows the syntax supported by the style files.
We encourage you to use the natbib styles.
You can use the command \verb|\citet| (cite in text) to get ``author (year)'' citations, like this citation to a paper by \citet{Gusfield:97}.
You can use the command \verb|\citep| (cite in parentheses) to get ``(author, year)'' citations \citep{Gusfield:97}.
You can use the command \verb|\citealp| (alternative cite without parentheses) to get ``author, year'' citations, which is useful for using citations within parentheses (e.g. \citealp{Gusfield:97}).

A possessive citation can be made with the command \verb|\citeposs|.
This is not a standard natbib command, so it is generally not compatible
with other style files.

\subsection{References}

\nocite{Ando2005,andrew2007scalable,rasooli-tetrault-2015}

The \LaTeX{} and Bib\TeX{} style files provided roughly follow the American Psychological Association format.
If your own bib file is named \texttt{custom.bib}, then placing the following before any appendices in your \LaTeX{} file will generate the references section for you:
\begin{quote}
  \begin{verbatim}
\bibliography{references}
\end{verbatim}
\end{quote}

You can obtain the complete ACL Anthology as a Bib\TeX{} file from \url{https://aclweb.org/anthology/anthology.bib.gz}.
To include both the Anthology and your own .bib file, use the following instead of the above.
\begin{quote}
  \begin{verbatim}
\bibliography{anthology,references}
\end{verbatim}
\end{quote}

Please see Section~\ref{sec:bibtex} for information on preparing Bib\TeX{} files.

\subsection{Equations}

An example equation is shown below:
\begin{equation}
  \label{eq:example}
  A = \pi r^2
\end{equation}

Labels for equation numbers, sections, subsections, figures and tables
are all defined with the \verb|\label{label}| command and cross references
to them are made with the \verb|\ref{label}| command.

This an example cross-reference to Equation~\ref{eq:example}.

\subsection{Appendices}

Use \verb|\appendix| before any appendix section to switch the section numbering over to letters. See Appendix~\ref{sec:appendix} for an example.

\section{Bib\TeX{} Files}
\label{sec:bibtex}

Unicode cannot be used in Bib\TeX{} entries, and some ways of typing special characters can disrupt Bib\TeX's alphabetization. The recommended way of typing special characters is shown in Table~\ref{tab:accents}.

Please ensure that Bib\TeX{} records contain DOIs or URLs when possible, and for all the ACL materials that you reference.
Use the \verb|doi| field for DOIs and the \verb|url| field for URLs.
If a Bib\TeX{} entry has a URL or DOI field, the paper title in the references section will appear as a hyperlink to the paper, using the hyperref \LaTeX{} package.

\section*{Limitations}

Since December 2023, a "Limitations" section has been required for all papers submitted to ACL Rolling Review (ARR). This section should be placed at the end of the paper, before the references. The "Limitations" section (along with, optionally, a section for ethical considerations) may be up to one page and will not count toward the final page limit. Note that these files may be used by venues that do not rely on ARR so it is recommended to verify the requirement of a "Limitations" section and other criteria with the venue in question.

\section*{Acknowledgments}

This document has been adapted
by Steven Bethard, Ryan Cotterell and Rui Yan
from the instructions for earlier ACL and NAACL proceedings, including those for
ACL 2019 by Douwe Kiela and Ivan Vuli\'{c},
NAACL 2019 by Stephanie Lukin and Alla Roskovskaya,
ACL 2018 by Shay Cohen, Kevin Gimpel, and Wei Lu,
NAACL 2018 by Margaret Mitchell and Stephanie Lukin,
Bib\TeX{} suggestions for (NA)ACL 2017/2018 from Jason Eisner,
ACL 2017 by Dan Gildea and Min-Yen Kan,
NAACL 2017 by Margaret Mitchell,
ACL 2012 by Maggie Li and Michael White,
ACL 2010 by Jing-Shin Chang and Philipp Koehn,
ACL 2008 by Johanna D. Moore, Simone Teufel, James Allan, and Sadaoki Furui,
ACL 2005 by Hwee Tou Ng and Kemal Oflazer,
ACL 2002 by Eugene Charniak and Dekang Lin,
and earlier ACL and EACL formats written by several people, including
John Chen, Henry S. Thompson and Donald Walker.
Additional elements were taken from the formatting instructions of the \emph{International Joint Conference on Artificial Intelligence} and the \emph{Conference on Computer Vision and Pattern Recognition}.

% Bibliography entries for the entire Anthology, followed by custom entries
%\bibliography{anthology,custom}
% Custom bibliography entries only
\bibliography{references}

\appendix

\section{Example Appendix}
\label{sec:appendix}

This is an appendix.

\end{document}
