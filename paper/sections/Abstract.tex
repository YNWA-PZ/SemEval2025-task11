\begin{abstract}
    This paper investigates multilingual emotion classification across three tasks: binary classification, intensity estimation, and cross-lingual emotion detection. To address challenges posed by linguistic diversity and limited annotated data, we explore a range of deep learning approaches, including transformer-based embeddings and traditional classifiers. Following extensive experimentation, language-specific embedding models were selected as the final approach due to their superior capability to capture linguistic and cultural nuances. Evaluations on both high- and low-resource languages demonstrate that this method yields strong performance, achieving competitive macro-average F1 scores across tasks. Notably, in the cross-lingual detection task, our approach secured first-place rankings in Oromo, Tigrinya, and Kinyarwanda, driven by the integration of advanced preprocessing techniques and tailored language modeling. Despite these advances, challenges persist due to data scarcity in underrepresented languages and the inherent complexity of emotional expression. This study underscores the importance of developing robust, language-aware emotion recognition systems and highlights future directions, including the expansion of multilingual datasets and continued refinement of modeling techniques.
\end{abstract}