\section{Introduction}
The analysis and processing of emotions from textual data have become crucial in understanding human communication across different languages and cultures. This study focuses on the detection and classification of emotions across diverse linguistic contexts, spanning regions from South America to East Asia. Our objective is to categorize emotions into key dimensions, namely sadness, anger, fear, disgust, joy, and surprise, while considering cross-lingual variations and linguistic complexities.

To address these challenges, we structure our study into three distinct tracks: (1) Track A involves binary emotion classification, determining whether a given text expresses a particular emotion; (2) Track B measures the intensity of emotions on a scale from 0 to 3, enabling a more granular understanding of emotional expressions; and (3) Track C explores cross-lingual emotion detection, facilitating insights into emotional patterns across different languages.

Understanding emotions based on textual data plays a pivotal role in various applications, including social media analysis, behavioral research, and the study of emotions' influence on social interactions. Our work contributes to the development of robust emotion recognition systems, enabling better comprehension of multilingual emotional expressions and their implications in computational linguistics.

Despite the significant advancements in emotion classification, several challenges persist. Some languages exhibit highly complex grammatical structures, making it difficult to train effective models. Additionally, the classification of emotions in low-resource languages is hindered by data scarcity and syntactic intricacies. Furthermore, certain machine learning models demonstrate suboptimal performance when applied to multilingual emotion classification, necessitating the development of novel techniques to enhance model adaptability and generalization.

To address these limitations, we present a comprehensive analysis of state-of-the-art methodologies and evaluate their effectiveness across multiple languages. Our findings highlight the critical role of innovative preprocessing techniques, domain adaptation strategies, and transfer learning in improving multilingual emotion classification.

All code implementations, including the models and experimental setups employed in this study, are publicly available on GitHub:\footnote{\url{https://github.com/YNWA-PZ/SemEval2025-task11}}. This repository provides full documentation of our methodologies, experimental results, and final model architectures.